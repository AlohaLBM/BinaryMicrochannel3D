\documentclass{CFD2011}
\usepackage{CFD2011}

%%%%%%%%%%%%%%%%%%%%%%%%%%%%%%%%%%%%%%%%%%%%%%%%%%%%%%%%%%%%%%%%%%%%%%%%%
%This template is created for complying with the author instructions for the 
%8th International Conference on CFD in Oil & Gas, Metallurgical and Process Industries
%hosted by SINTEF/NTNU, Trondheim Norway
%21-23 June 2011
%
%Sverre G. Johnsen (sverre.g.johnsen@sintef.no)
%SINTEF Materials and Chemistry


%Required files:
%   ExampleFile.tex
%   xampleFile.nls
%   CFD2011.bst
%   CFD2011.cls
%   CFD2011.sty
%   References.bib

%Example-specific files:
%   figure.eps
%   Table.tex
%%%%%%%%%%%%%%%%%%%%%%%%%%%%%%%%%%%%%%%%%%%%%%%%%%%%%%%%%%%%%%%%%%%%%%%%%




\title{Industrial Applications of CFD}
\paperID{CFD11-52}
\author{Ola}{Nordmann} %{forename}{surname}
\presenting  %the previous author is presenting the paper (name becomes underlined)
\address{SINTEF Materials and Chemistry, 7465 Trondheim, NORWAY}%affiliation of the previous author
\email{ola.nordmann@sintef.no}%e-mail address of the previous author
\author{Zhi}{L. Xie}
\address{NTNU Department of Physics, 7491 Trondheim, NORWAY}
\email{lin.xie@ntnu.no}
%\author{Sverre}{G. Johnsen}
%\address{SINTEF Materials and Chemistry, 7465 Trondheim, NORWAY}
%\email{sverre.g.johnsen@sintef.no}



\begin{document}
\maketitle  %create the title page
\headers   %create the page headers and footers


\abstract{
  This file is an example \LaTeX file for submission to CFD2011.  A
  limit of 10 pages applies.
}
\keywords{
  CFD, hydrodynamics, chemical reactors
}
\normalfont\normalsize



%\section{Nomenclature}
A complete list of symbols used, with dimensions, is required.
%the nomenclature needs to be entered into the file ExampleFile.nls
\printnomenclature[0.7cm]  
\vskip .1em


\section{Introduction}
The introduction goes here.


\newpage



\section{Model Description}
You should give a thorough description of your model.
\vspace{4cm}
\subsection{Example of Subheading}
Here is how to produce a numbered equation under a second level
heading \cite{James1988}.
\vspace{2cm}\\
\emph{Continuity equation}
\begin{equation}
  \frac{\partial \rho_G}{\partial t}+\nabla\left(\rho_G\mathbf
    u\right)=0
\end{equation}

\subsubsection*{Example of Sub-subheading}
This is how \cite{Luke1988} produced an unnumbered equation under a
third level heading.
\begin{equation}
  \mathbf J=\sigma(\mathbf E+\mathbf u\times\mathbf B)
\end{equation}
\newpage

\InsFig{figure}{Schematic diagram of geometry.}{label1}
\InsFigRot{figure}{Rotated schematic diagram of geometry.}{label2}


\section{Results}
The results of using the \LaTeX template is a great looking paper.
In Figures \ref{fig:label1} and \ref{fig:label2} it can be seen how figures are easily included.
In Table \ref{tab:label} it is seen how we can include a table.
The table is constructed in the file table.tex, where also the table caption and label are defined.


\newpage
\InsTab{Table}


\section{Conclusion}
The conclusions are:
\begin{enumerate}
  \item Trondheim is a nice city.
  \item CFD is great fun, and useful too.
\end{enumerate}

\newpage


%% %---------------------------
%% %BIBLIOGRAPHY
%% %---------------------------
%The bibliography is created using BiBTeX
\bibliographystyle{CFD2011}
\bibliography{References}   


\newpage
\section{Appendix A}
List of animation files:
s10.avi Motion of the spherical particles.
v10.mped Flow field around each particle.


\end{document}
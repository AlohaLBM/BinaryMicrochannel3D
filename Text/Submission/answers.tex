\documentclass{article}
\usepackage{amsmath,amssymb}
\usepackage{color}
\usepackage{bm}
\usepackage[sort&compress]{natbib}
\usepackage[normalem]{ulem}	% Part of the standard distribution


\begin{document}
It is our pleasure to address the reviewers' comments. Please consider our answers below. All corrections in the manuscript are marked by red.
\section{Reviewer I}
Question and Answers:

\textbf{Question:} When Re is very small, the authors neglect the density ratio of gas and liquid phase. I think more
reasons are still needed to explain the uniform density, especially on the interface. One of the
challenges for free-energy LB model is how to meet the Galilean invariance because of the density
gradient on the interface.

\textbf{Answer:}  We added additional text explaining the choice of the binary-liquid model towards the simulation
of the Bretheron-Taylor phenomenon (see lines 99-112). The Galilean invariance is usually fixed by
third-order velocity terms in the equilibrium, see corresponding work of \citet{qian-galilean}.
However, the velocities in the current simulations are of order $0.001-0.01$ and we can neglect the
non-Galilean invariance.

\textbf{Question:} In Figures 7 and 8, it's better to give the reported curves by Liu and Wang and Hazel and Heil, so
that it's easy to find the agreement is well or not.

\textbf{Answer:} We added additional figures. 

\textbf{Question:} In Figure 9, the variation of bubble radii in the range of capillary number 0.05<=Ca<=0.6 is
still different from the results of Hazel and Heil and of Liu and Wang. The critical capillary
number of Liu and Wang is close to 0.8 and the critical capillary number of Hazel and Heil is close
to 0.04, but the value of the authors is 0.1. The authors should analyze the reason of the
difference.

\textbf{Answer:} The reviewer is right that the critical capillary number of Liu and Wang is close
to 0.8 as it can be seen from their digitized numbers. However, Liu and Wang state in their paper
that the critical capillary number is 0.1. That means they used weaker criterion for the axial
radius to be equal to the diagonal radius. Therefore we consider that the critical number for
transition is between 0.04 and 0.1. The criterion is not well defined and depends on the flow
conditions. For example, Hazel and Heil simulate the air finger propagation with the bubble length
of more than 10 diameters of the channel. In our case, we simulate the bubble train where each
bubble has a certain variation of the film thickness over the bubble length. We chose to measure the
film thickness in the middle of the bubble. There is an assumption that the bubble length is long
enough (3 diameters) for thickness to coincide with the air finger film thicknesses of the infinite
length \cite{cerro-space}. However, in the case of smaller capillary the film thickness variation
is quite large. Thus, more defined criteria should be given in papers in order to validate our
results. Moreover, some formulations like the propagation of the semiinfinite air finger is not
achievable due to computational resources constraints.

\textbf{Question:} In Figure 10, there is a deviation of the relative velocity between the present and published data.
Can the authors extend the range of capillary numbers as conducted by Liu and Wang?

\textbf{Answer:} Unfortunately that requires months with the uniform grids formulations of the
lattice Boltzmann method or some grid refinement needed. We didn't have aim to go towards smaller
capillary numbers as they nicely covered by original work of \citet{bretherton}.

\textbf{Question:} Check some errors, e.g., on Page 7 Line 137 "0.05>=Ca<=6", on Page 17 Line 300, "82×82×1500 and
82×82", and on Page 24 References 18 and 20.

\textbf{Answer:} Thank you for pointing that. All the errors are corrected. Reference 20 is
available online and is not published in journals.

\section{Reviewer II}
Questions and Answers:

\textbf{Question:} Generally speaking, the motion of a drop in a channel is a function of channel shape, drop size,
ratio between driving force and surface tension, density ratio, and viscosity ratio.  For the flow
of a long gas bubble (Taylor-Bretherton) the density and viscosity ratios are nearly zero, making it
possible to fit the film thickness as a function of a single dimensionless number Ca.  In the
problem studied in this paper, the bubble is long, the density ratio is unity and the viscosity
ratio is 0.1 (gas over liquid).  As such, while the film thickness in this system is also reduced to
a function of the Capillary number, I do not think that the system considered by the authors is a
Taylor-Bretherton system, as the density and viscosity ratios are different. The authors used the sensitivity of the film thickness on Re to justify that the sensitivity on
density ratio is low (the first paragraph of Page 6).  However, I do not see the one-to-one
connection between Re and the density ratio.  Are there any direct numerical/experimental evidence
that density ratio has small effect on the film thickness, or evidence that Re is solely (or nearly
solely) determined by the density ratio?

\textbf{Answer:}  The reviewer is right about not a full compliance with the classical
Bretherton/Taylor problem. We added more explanations (lines 99-112). The system is governed with
the non-dimensional numbers. Mainly, apart from geometrical dimensions the system can be fully
descisribed by the following set of parameters: Re, Ca and Bo (Eo), where the Bond number Bo is
taken into the account density ratio and is mainly involved in problems for a bubble change.
However, the Bond number in simulations and real experimental setups is low. That's why we consider
it as a non-leading non-dimensional parameter. 
 
\textbf{Question:} Aside from this critique, it looks like the numerical resolution needs some clarifications.  While
some numbers and discussions were provided, they seem to be in conflict with each other.  For
example, in the discussion given in Page 8 and top of Page 9, the authors stated that the minimum
resolution needed to resolve the film is 600x600x9000, and, if a quarter of the domain is simulated,
a resolution of 300x300x9000 is needed.  However, the domain sizes presented later are all smaller,
e.g. 100x100x1500, 160x160x1500, 160x160x2400, 200x200x3000 (Page 12).  How were these grid
resolutions selected, and what is the effect of grid resolution on the result?

\textbf{Answer:} As an example, the corresponding grid $600\times600\times9000$ was taken to resolve
$\delta=0.49 H_{\mathrm{eff}}$ for $Ca=0.03$ from reported simulations of \citet{heil-threedim}.
However, luckily for us for current simulations the value of the film thickness was higher
than $0.49 H_{\mathrm{eff}}$. That's why the computational resources demand is lower and we used
smaller grids. The simulation for $600\times 600 \times 9000$ is too much computationally demanding.
To give account for numbers one simulation with the grid $200\times200\times 3000$ is obtained after
$24$ days on $32$ CPUs. The values of these grids were chosen to resolve the film thickness for the
smallest capillary number (film thickness is twice bigger than the interface thickness). We've done
inextensive grid resolution study (see Fig. 9) to support our results.  We clarified
this issue on lines 168-171. 

\textbf{Question:} A couple of minor issues are listed below:
Page 4, figure 1 - I suppose the vertical axis is $r_d$ and $r_h$ normalized by $H_eff$?  The axis
label perhaps should be clarified.\\
Page 5, line 18 - missing "from" between "capillary number" and "simulations".\\
Page 7, line 19 - "0.05 >= Ca <= 6" should be "0.05 <= Ca <= 6".\\
Page 10 - On the lattice Boltzmann free energy model, what are the boundary conditions for f and g
populations?  Were they simple bounce-back?  Is contact angle implemented in the method?\\
Page 12, line 13 - "k = A = 0.004, 0.04".  I do not understand this notation - does this mean k is
always equal to A and their values should be between 0.004 and 0.04?\\
Page 17, line 17-19 - The text says that Figure 9 contains data with different initializations and
different grids.  However, the distinction between data sets cannot be viewed from the figure. 
Perhaps different symbols should be used so that the readers can tell one data set from another?\\
Page 17, line 40 - "differencies" should be "differences".\\
Page 19, line 25 - ". the film thickness as twice the interface thickness".  Please see the previous
comment on grid resolution.\\
Page 20, line 19 - Again, the notation used for the range of A and k is rather unusual.\\
Page 29, line 46 - ". relaxed after to the axis radius" - perhaps a typo in the sentence?

\textbf{Answer:} Thank you for improving the article. All the corrections are taken into the
account. We also added more explanations to the text to address the following questions:
\begin{itemize}
 \item For $f$ and $g$ populations the wall was simulated using the simple bounce-back rule. We
showed before that results do not depend on the wettability of the wall
\cite{kuzmin-binary2d}. Thus, the walls were chosen with neutral wettability. 
 \item We chose $k$ and $A$ equal to each other as this choice conserves the film thickness ($5$
grid spacings). $k=A=0.04$ is taken for most simulations. $k=A=0.004$ is taken for simulation with 
$Ca=6$. Usually for stability issues the choice of $k=A$ is a good one. The values of $k$ and $A$
can vary in different limits. However, the stability is drastically reduced for $k=A=0.25$ or if
$k=A \ll 1$. The corresponding text is added (lines 219-221).
\item Figure 9 shows the results grid independency. It's exactly the reason why we didn't
distinguish different initializations and grid numbers as proper designed simulations should not
depend on it. Corresponding text is added (lines 316-317).
\end{itemize}

\textbf{Question:} Finally, I have a question on how the initial distributions of f and g were determined?  Did the
simulation start with uniform distributions of f and g populations and they "phase separate" to form
the bubble?

\textbf{Answer:} Initial distributions $f$ and $g$ were taken through equilibrium distribution
functions, which were constructed from initial phase and density fields. Density was taken uniform
with the value $1$ across the domain. Bubble was initialized as the rectangular parallelepiped in
the middle domain. After that the system evolves with time and interplay between surface tension
and viscous forces form the bubble. It was shown \cite{kuzmin-binary2d} that the final steady-state
result does not depend on the initialization. 
 
\section{Reviewer III}:

Questions and Answers:

\textbf{Question:} Check some spelling errors, also in the abstract;

\textbf{Answer:} We thoroughly corrected all found mistakes.

\textbf{Question:}  Improve Figure quality: Figs. 4 and above all 5 are hard to read and understand. The ticks should
be the same height, those in Fig. 5 are hardly visible.

\textbf{Answer:} We constructed figures with the same size ticks, title and labels. For better
readability we rearranged figures to be of larger size.

\textbf{Question:}  The bibliography should be improved: when introducing LB, I think that more hints should be given
to the applications, for instance to the case of magnetically-driven vaporization in ferrofluids
studies or regarding multiphase approach based on pseudopotentials.

\textbf{Answer:} We expanded the introduction for industry applications and discussion of
multiphase models (lines 117-123).

\textbf{Question:} Did the Authors perform any comparison with pseudopotential multiphase approaches? I am
especially referring to recent advances of pseudopotentials, such as the case of doubly-attractive
potentials and to recent papers regarding LB multiphase simulation across scales.

\textbf{Answer:} Unfortunately, to our best knowledge there is only one study of the pseudopotential
model towards the Bretherton-Taylor phenomenon. The study is in 2D and cannot be applicable to this
case. However, we outlined the approaches for pseudopotential models (lines 117-123) and all the paper findings are applicable for pseudopotential models as well.

\textbf{Question:} Could the Authors give more hints on the nature of the walls? Are they hydrophilic or hydrophobic and how do the Authors implement such a nature?

\textbf{Answer:} The walls are neutral. It was shown previously \cite{kuzmin-binary2d} that for the Bretherton phenomenon when the film thickness is resolved the nature of the walls is not involved.
\bibliographystyle{unsrtnat}
\bibliography{paper}

\end{document}
